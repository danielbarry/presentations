%%%%%%%%%%%%%%%%%%%%%%%%%%%%%%%%%%%%%%%
% presentation.tex
%
% This presentation has been designed to install the basics of Git into
% Computer Science students at the University of Hertfordshire.
%
% @author B[]
%%%%%%%%%%%%%%%%%%%%%%%%%%%%%%%%%%%%%%%

\documentclass{beamer}

\usepackage{color}
\usepackage{listings}
\PassOptionsToPackage{hyphens}{url}\usepackage{hyperref}

\usetheme{metropolis}

\definecolor{Blue}{rgb}{0.0, 0.0, 0.8}
\definecolor{DarkGreen}{rgb}{0.0, 0.5, 0.0}
\definecolor{DarkGray}{rgb}{0.5, 0.5, 0.5}
\definecolor{Gray}{rgb}{0.8, 0.8, 0.8}
\definecolor{Orange}{rgb}{0.8, 0.5, 0.0}
\definecolor{White}{rgb}{1.0, 1.0, 1.0}

\lstset{
  basicstyle=\ttfamily\footnotesize,
  breakatwhitespace=false,
  breaklines=true,
  captionpos=b,
  commentstyle=\color{DarkGreen},
  escapeinside={\%*}{*)},
  extendedchars=true,
  frameshape={RYR}{Y}{Y}{RYR},
  keepspaces=true,
  keywordstyle=\color{Blue},
  numbers=left,
  numbersep=6pt,
  numberstyle=\tiny\color{DarkGray},
  rulecolor=\color{Gray},
  showspaces=false,
  showstringspaces=false,
  showtabs=false,
  stepnumber=1,
  stringstyle=\color{Orange},
  tabsize=2
}

\title{Git Basics}
\date{November, 2016}
\author{Daniel Barry}
\institute{University of Hertfordshire}

\begin{document}
  \maketitle
  %%%%%%%%%%%%%%%%%%%%%%%%%%%%%%%%%%%%%%%
  % Introduction
  %%%%%%%%%%%%%%%%%%%%%%%%%%%%%%%%%%%%%%%
  \section{Introduction}
  \begin{frame}{What is Git?}
    \textbf{TODO:} Write this section.
  \end{frame}
  \begin{frame}{Why use Git?}
    \textbf{TODO:} Write this section.
  \end{frame}
  %%%%%%%%%%%%%%%%%%%%%%%%%%%%%%%%%%%%%%%
  % Getting Started
  %%%%%%%%%%%%%%%%%%%%%%%%%%%%%%%%%%%%%%%
  \section{Getting Started}
  \begin{frame}{Requirements}
    \begin{itemize}
      \item \texttt{git} - The repository program
      \item \texttt{git-gui} - Graphical representation
    \end{itemize}
  \end{frame}
  \begin{frame}[fragile=singleslide]{Setup an Environment}
    Create the working directory and navigate to it
    \begin{lstlisting}[language=bash]
$ mkdir wrk_dir
$ cd wrk_dir
    \end{lstlisting}
    Initialise the repository in the working directory
    \begin{lstlisting}[language=bash]
$ git init
    \end{lstlisting}
  \end{frame}
  %%%%%%%%%%%%%%%%%%%%%%%%%%%%%%%%%%%%%%%
  % Workflow
  %%%%%%%%%%%%%%%%%%%%%%%%%%%%%%%%%%%%%%%
  \section{Workflow}
  \begin{frame}{Adding Files}
    \textbf{TODO:} Write this section.
  \end{frame}
  \begin{frame}{Visually Adding Files}
    \textbf{TODO:} Write this section.
  \end{frame}
  \begin{frame}{Commiting \& Pushing}
    \textbf{TODO:} Write this section.
  \end{frame}
  \begin{frame}{Stashing}
    \textbf{TODO:} Write this section.
  \end{frame}
  \begin{frame}{Pulling}
    \textbf{TODO:} Write this section.
  \end{frame}
  \begin{frame}{Feature Branch}
    \textbf{TODO:} Write this section.
  \end{frame}
  %%%%%%%%%%%%%%%%%%%%%%%%%%%%%%%%%%%%%%%
  % Something Went Wrong
  %%%%%%%%%%%%%%%%%%%%%%%%%%%%%%%%%%%%%%%
  \section{Something Went Wrong}
  \begin{frame}{Undo-ing Changes}
    \textbf{TODO:} Write this section.
  \end{frame}
  \begin{frame}{Undo-ing Commits}
    \textbf{TODO:} Write this section.
  \end{frame}
  %%%%%%%%%%%%%%%%%%%%%%%%%%%%%%%%%%%%%%%
  % Advanced
  %%%%%%%%%%%%%%%%%%%%%%%%%%%%%%%%%%%%%%%
  \section{Advanced}
  \begin{frame}{Re-Writing History}
    \textbf{TODO:} Write this section.
  \end{frame}
  %%%%%%%%%%%%%%%%%%%%%%%%%%%%%%%%%%%%%%%
  % Conclusion
  %%%%%%%%%%%%%%%%%%%%%%%%%%%%%%%%%%%%%%%
  \section{Conclusion}
  \begin{frame}{About Presentation}
    \begin{itemize}
      \item Content - \url{https://github.com/danielbarry/presentations}
      \item Presentation - \url{http://www.latex-project.org}
      \item Theme - \url{https://github.com/matze/mtheme}
    \end{itemize}
  \end{frame}
\end{document}
